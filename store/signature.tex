\documentclass[a5paper]{article}
\usepackage[utf8]{inputenc}
\usepackage[spanish]{babel}

\usepackage[pass]{geometry}
\usepackage{tikz}
\usepackage{hyperref}

\pdfpageheight=210mm
\pdfpagewidth=148mm

\begin{document}

\begin{titlepage}

%\vspace{-3cm}

\textsl{\Huge Booklet Machine}

\begin{center}
\textit{de}
\end{center}

\begin{flushright}
\textsc{\Huge Miguel Cruces}

\end{flushright}

\vfill



\textit{Este es un proyecto independiente que nació por el amor a los libros, programado en} \textsf{Python} \textit{durante los ratos libres mientras estudiaba en la} \textsc{Facultad de Física, USC}. 
 
\textit{La motivación para desarrollar este trabajo surgió de la necesidad de encuadernar a mano libros escritos por uno mismo, ya que en el momento no existía ningún otro trabajo satisfactorio para mí por la red. Entonces, decidí crear el mío propio y compartirlo, para que más personas lo puedan disfrutar y me ayuden a encontrar errores o posibles mejoras.}

%\begin{center}
%{\huge $\Psi$}
%\end{center}

\begin{flushright}
\textit{Gracias por usar} \textsl{Booklet Machine}.

Paz.

\vspace{0.25cm}
\end{flushright}


\vfill

\begin{center}
\begin{tikzpicture}
\draw (0,0) rectangle (1.618*3,1*3);
\draw (1*3,0) rectangle (1.618*3,1*3);
\draw (1*3,0) rectangle (1.618*3,1);
\draw (1*3,0) rectangle (3+0.618,1);
\draw (1*3,0.618) rectangle (3.618,1);
\end{tikzpicture}
\end{center}

\vspace*{-2cm}

\end{titlepage}

\pagebreak

\begin{titlepage}
\vfill

\begin{center}
Con el paquete \textsf{PyPDF2} de \textsc{Matthew Stamy}, en \textsf{GitHub}: \url{https://github.com/mstamy2/PyPDF2}

\vspace{2cm}

\textsf{\textbf{PyPDF2}}

\vspace{0.5cm}

{\footnotesize \textsf{PyPDF2} is a \textsf{pure-ython PDF library} capable of splitting,}

{\footnotesize merging together, cropping, and transforming the pages of PDF files.}

{\footnotesize It can also add custom data, viewing options, and passwords}

{\footnotesize to PDF files. It can retrieve text and metadata from PDFs as well as merge entire files together.}

\end{center}

\vspace{5cm}

\textbf{Documentación en:}

\url{https://pythonhosted.org/PyPDF2/}

\vfill
\end{titlepage}

\end{document}